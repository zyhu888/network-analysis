\documentclass[12pt]{article}

%% preamble: Keep it clean; only include those you need
\usepackage{amsmath}
\usepackage[margin = 1in]{geometry}
\usepackage{graphicx}
\usepackage{booktabs}
\usepackage{natbib}

%% for double spacing
\usepackage{setspace}

% for space filling
\usepackage{lipsum}
% highlighting hyper links
\usepackage[colorlinks=true, citecolor=blue]{hyperref}

\usepackage{indentfirst}

%% meta data

\title{Literature Review}
\author{Qingkai Dong\\
}

\begin{document}
\maketitle

\begin{abstract}

\end{abstract}

\doublespacing

\section{Discussions}

\subsection{Problem Description}

In the design phase, we predetermine the number of events $n_{F}$ for the final
analysis and plan to perform the interim analysis when $n_{I}$ events are
observed. Define information fraction as $t=n_{I}/n_{F}$ and incremental sample
size $\tilde{n}_{F}=n_{F}-n_{I}$.  Given the interim data, we want to calculate
the Conditional Power(CP) for the final analysis, which depends on the value of
the interim analysis statistic $Z_{I}$, the sample size $n_{I}$ and $n_{F}$, the
information fraction $t$ and the true effect $\theta$. Specifically,
\citep{teng2020seamless} frames this problem under proportional hazards
assumption.  Assume $h_{1}(t|X) = h_{0}(t) e^{X(-\theta)}$ where $X$ is the
binary treatment variable. Assume time-to-event variable $Y \sim Exp(\lambda)$
where $f_{Y}(y)=\lambda e^{-\lambda y}$, hazards function
$h_{Y}(y)=\lambda$. Let $\lambda=e^{-X\theta}$. In this case, $h_{0}(t)=1$,
$HR=\frac{\lambda_{(X=1)}}{\lambda_{(X=0)}}=e^{-\theta}$ and
$\theta=-\log(HR)$. Thus, $HR < 1$ ($\theta > 0$) favors the treatment
arm. Assume the estimate of treatment effect $\hat{\theta}$ is asymptotically
normal: $\hat{\theta} \sim N(\theta, \frac{4}{n})$. It holds for $n=n_{I}$ and
$\tilde{n}_{F}$.

The two-side test is: $H_0: \theta=0$ vs. $H_1: \theta \neq 0$. The interim
logrank test statistic $Z_{I} = \hat{\theta}_{I} / \sqrt{\frac{4}{n_{I}}}$. The
increment test statistic is $\tilde{Z}_{F} = \tilde{\theta}_{F} /
\sqrt{\frac{4}{n_{F}-n_{I}}}$ which is asymptotically normal $N(\theta /
\sqrt{\frac{4}{n_{F}-n_{I}}}, 1)$. The
final test statistic is $Z_{F} = \hat{\theta}_{I} / \sqrt{\frac{4}{n_{I}}} =
\sqrt{t}Z_{I} + \sqrt{1-t}\tilde{Z}_{F}$.


The unconditional power of the final analysis is
$P(Z_{F}>Z_{1-\frac{\alpha_{F}}{2}}|\theta=\theta_{D})$ where $\theta_{D}$ is
pre-specified at the design phase. The conditional power of the final analysis
is $P(Z_{F}>Z_{1-\frac{\alpha_{F}}{2}} | Z_{I})$.

\begin{align*} CP = P(Z_{F}>Z_{1-\frac{\alpha_{F}}{2}} | Z_{I}) &= P(\sqrt{t}Z_{I} +
\sqrt{1-t}\tilde{Z}_{F}>Z_{1-\frac{\alpha_{F}}{2}} | Z_{I}) \\ 
&= P\left(
\tilde{Z}_{F} > \frac{Z_{1-\frac{\alpha_{F}}{2}} - \sqrt{t}\hat{\theta}_{I} /
\sqrt{\frac{4}{n_{I}}}}{\sqrt{1-t}} \right) \\ &= 1 -\Phi\left(
\frac{Z_{1-\frac{\alpha_{F}}{2}} - \sqrt{t}\hat{\theta}_{I} /
\sqrt{\frac{4}{n_{I}}}}{\sqrt{1-t}} - \frac{\theta}{\sqrt{\frac{4}{n_{F}-n_{I}}}}\right). \\
 \end{align*}
By plugging in $\hat{\theta}_{I}$ to replace $\theta$, we have an estimate of CP
$$\hat{CP} =  1 -\Phi\left(
 \frac{Z_{1-\frac{\alpha_{F}}{2}}}{\sqrt{1-t}} -
\frac{\sqrt{t}\hat{\theta}_{I}}{\sqrt{1-t}\sqrt{\frac{4}{n_{I}}}}
-\frac{\hat{\theta}_{I}}{\sqrt{\frac{4}{n_{F}-n_{I}}}}\right).$$
Then we have $$Z_{1-CP}=\frac{Z_{1-\frac{\alpha_{F}}{2}} -
\sqrt{t}\hat{\theta}_{I} / \sqrt{\frac{4}{n_{I}}}}{\sqrt{1-t}} -
\frac{\hat{\theta}_{I}}{\sqrt{\frac{4}{n_{F}-n_{I}}}}$$ and $$n_{F}^{new} =
n_{I} + 4\left(\frac{Z_{1-\frac{\alpha_F}{2}}}{\sqrt{1-t}} -
\frac{\sqrt{t}\sqrt{n_{I}}}{2\sqrt{1-t}}\hat{\theta}_{I} - Z_{1-CP}\right) /
\hat{\theta}_{I}^2.$$

Here the interim analysis statistic $Z_{I}$ and the information fraction $t$
are considered fixed. In this case, given an estimate of the true
effect $\hat{\theta}$, we can make decisions based on the value of CP:
fail, increase the the sample size and continue the trial, keep the
sample size and continue the trial or success. In this way, giving an
inaccurate estimate of the true effect will lead to wrong interim
decisions.

Usually $\hat{\theta}$ is calculated based on $n_{I}$ data from the
interim analysis and only treatment is considered as covariate. However, in addition to the event times and features
of the $n_{I}$ observations that have already had events, we have the
features of the $N-n_{I}$ observations that recruited but did not have
events at interim analysis. $N$ is the number of recruitments. The
problem is how to provide a $\hat{\theta}$ that makes CP more
accurate.

\subsection{Simulations}

We illustrate the problem with a simple data setting and simulation. Denote the treatment effect as $\theta$. Following \citep{teng2020seamless}, assume $h_{1}(i)=h_{0}(i)e^{-\theta}$, that is, $\theta = -\log(HR)$. Assume time-to-event variable $Y \sim Exp(\lambda)$ where $f_{Y}(y)=\lambda e^{-\lambda y}$, hazards function $h_{Y}(y)=\lambda$. Let $\lambda=e^{-X\theta}$. In this case, $HR=\frac{\lambda_{(X=1)}}{\lambda_{(X=0)}}=e^{-\theta}$.

Let $N=300$, $n_2 = 200$ and $n_1=100$. Let $\theta_{true}=0.4$. Define model 1 as an AFT model with only treatment variable as covariate. Suppose there are three other covariates $X_2$, $X_3$ and $X_4$ so that we fit model 2 with four covariates. We compute CP based on the formula in \citep{teng2020seamless} and take the mean of 1000 replicates. Here are the boxplots.

Questions: 

- In the clinical trial context, is fitting an AFT model a common way to estimate $\theta$?

- By fitting AFT model, we have already considered informations from censored observations. What we can do further is to find a method that can optimize the estimation efficiency of $\theta$.
\section{Review}

Review \citep{qi2019predicting}.

\subsection{Introduction}

Phase 3 and Phase 2 may have divergent results. This is due to the
biased prediction of Phase 3 results caused by multiple confounding
effects in Phase 2, inadequate model assumptions, and/or the population
shift between Phase 2 and Phase 3.

According to the results of a Phase 2 clinical trial, the treatment
effects can be significant. However, after a Phase 3 trial, people may
found the effect shrunk. This may be due to many reasons, such as
population shift, change of endpoints, change of treatment regime,
etc.. With the help of the proposed framework, the reason of
inconsistent treatment effects can be better examined.

Pharmacokinetic concentrations, or PK concentrations, refer to the
levels of a drug in the blood at various times after the drug has been
administered. Traditional population PK (PopPK) models can provide
accurate estimate to the average trough concentration for the whole
population. The PK relationship can be heterogeneous, highly
nonlinear, and impacted by many patient characteristics, so it is
often challenging for traditional models to accurately capture the
complex nonlinear relationship at subject level.

This paper proposed to predict Phase 3 trial results by using Phase 2
information and baseline characteristics of Phase 3
patients. 

\subsection{Data}

The outcomes are patients’ responses to an investigational treatment
measured by change from baseline of a physical function at Week
16. Patients’ baseline characteristics pre-specified in the primary and
exploratory analyses of Phase 2 are included as time-invariant
predictors. Dose level and Ctrough of each patient in the Phase 2
trial are time-variant predictors since they change week by week. We
also have baseline characteristics of Phase 3 patients.

\subsection{Models}

The target is to predict ITE(individual treatment effect) for patients 
in the Phase 3 when Phase 2 trial is finished. To achieve this, they 
build two models.

Ctrough model $\phi$ aims to predict the Ctrough(trough 
concentrations) for Phase 3
patients. Trained by using Phase 2 data. Trough refers to the lowest
concentration of a drug in the bloodstream, typically measured just
before the next dose is administered.

ITE model $f$ aims to predict ITE by using Ctrough and baseline
characteristics as input. It is trained using Phase 2 data.

Notations are as follows. Denote $\{c^{(2)}_i,
\tilde{x}^{(2)}_i\}_{i=1}^{k}$ as Phase 2 data, where $c^{(2)}_i$ is
the sequence of observed Ctrough of the i-th patient,
$\tilde{x}^{(2)}_i=\{x^{(2)}_i, d^{(2)}_i\}$, $x^{(2)}_i$ is the
$p$-dimensional characteristics, $d^{(2)}_i$ is the sequence of dose
levels assigned to the i-th patient. $k$ is the number of patients who
have observed Ctrough.

$\phi$ is a neural network model. The input of $\phi$ is
$\{\tilde{x}^{(2)}_i\}_{i=1}^{k}$ and the output is
$\{\hat{c}^{(2)}_i\}_{i=1}^{k}$. By training and validating Ctrough model
$\hat{\phi}$ on $\{c^{(2)}_i, \tilde{x}^{(2)}_i\}_{i=1}^{k}$, they use
it to predict $c^{(3)}_j$ by inputting $\tilde{x}^{(3)}_j$.

Next they train and validate ITE model $f$ which is also a neural
network. The training data is $\{y^{(2)}_i,
\Bar{x}^{(2)}_i\}_{i=1}^{n}$ where $\Bar{x}^{(2)}_i = \{x^{(2)}_i,
d^{(2)}_i, c^{(2)}_i\}$. For those patients without observed Ctrough,
their $c^{(2)}_i$ will be imputed by $\hat{\phi}(x^{(2)}_i,
d^{(2)}_i)$. $y^{(2)}_i$ is the $r$-dimensional response variable
consisting of the $r$ endpoints of the i-th patient in Phase 2. Here
they focus on one endpoint situation so that $y^{(2)}_i$ is a
scalar. Apply $\hat{f}$ to Phase 3 data, they get the predicted
response $\{\hat{y}^{(3)}_j\}_{j=1}^{m}$, that is, $\hat{y}^{(3)}_j =
\hat{f}(\Bar{x}^{(3)}_j) = \hat{f}(\{x^{(3)}_j, d^{(3)}_j,
\hat{c}^{(3)}_j\})$. The predicted response for patient who receives
placebo is $\tilde{y}^{(3)}_j = \hat{f}(\{x^{(3)}_j, 0, 0\})$. The ITE
for patient $j$ is $\hat{y}^{(3)}_j - \tilde{y}^{(3)}_j$ and the
average treatment effect is $\sum_{i=1}^{m}(\hat{y}^{(3)}_j -
\tilde{y}^{(3)}_j)/m$.

The proposed recurrent neural network is
\begin{align*}
  s_{i0} &= \sigma(T x_i + c) \\
  s_{it} &= \sigma(U
           z_{it} + W s_{i,t-1} + b), \quad t=1,...,T_i. \\
  \hat{y}_{it} &=
                 \sigma(V s_{it} + a) + P s_{i0}, \quad t=1,...,T_i.\\
\end{align*}
For Ctrough model, $x_i = x^{(2)}_i$ and $z_{it} =
d^{(2)}_{it}$. For ITE model, $x_i = x^{(2)}_i$ and $z_{it} =
(d^{(2)}_{it}, c^{(2)}_{it})$. Specifically, at each time point $t$,
we utilize information from current ($z_{it}$) and previous
($s_{i,t-1}$) to predict the current response ($\hat{y}_{it}$).

\subsection{Discussion}

\begin{itemize}
  \item The author didn't release code. They didn't mention the sample size.
  \item Overfitting.
  \item Inference.
\end{itemize}

\section{Simulations and ML+design Literature Review}

Present a simulation. Review literature on clinical trial design with
machine learning methods. 

\subsection{Simulation}

After meeting, we modify the description of the problem and conduct
new simulations.

In a clinical trial investigating whether treatment effects affect
time-to-event endpoints, we perform log rank test at the final
analysis using patients who have been observed events. For patients
who are censored at the interim analysis, we can fit a model to
predict their event times. In this way, at the interim analysis, we
can perform the second log rank test for all patients by combining
true event times for uncensored patients and predicted event times for
censored patients.

The following questions need to be answered. What data can we use to
predict event times for censored individuals? Under what circumstances
would the second log rank test be more powerful than the first? Can we
combine predicted event times with true event times directly?

The simulation settings are as follows.  Define $Y$ as the event time
which comes from exponential distribution with mean $e^{X \beta}$. We
enroll $N=300$ patients with seven covariates: treatment $\sim
Bin(0.5)$ and other six. The interim time is the time when we observe
$n_{int} = \frac{1}{3}N$ events. The final analysis time is when we
observe $n_{fin} = \frac{2}{3}N$ events. The recruitment time follows
$U(0,0.2)$.  We generate simulated data for 1000 times. Fix
$\alpha=0.05$.

The gray line is the power of using only $n_{int}$ patients at the
interim analysis. The green line is the power of using $n_{fin}$
patients at the final analysis. The blue line is the power of using
$N$ patients (suppose we know their true event time). The red line is
the power of using $n_{int}$ patients with true event time and
$(N-n_{int})$ patients with predicted event time at the interim
analysis.

\begin{figure}[!ht]
\centering
\includegraphics[width=0.6\textwidth]{p1.png}
\caption{Effect vs. Power}
\label{fig:effect}
\end{figure}

\subsection{Literature Review}

\citep{kavalci2023improving} used random forest and boosting method to
predict early termination probability of clinical trials and to
understand feature contributions driving this outcome. They use
420,268 clinical trial records and 24 fields was extracted from the
ct.gov registry.  \citep{schwager2021utilizing} designed a ML model
for stratifying patients, ultimately reducing the required number of
patients by increasing statistical power through cohort
homogeneity. From the Philips eICU Research Institute (eRI) database,
no less than 51,555 ARDS patients were extracted. As in the two papers
above, there is also a lot of work that utilizes large dataset like
electronic health records and genomic data to train machine learning
models.

\citep{romero2020statistical} proposed a deep learning model that
estimated the contributions of individual patients to a study’s
statistical power in epilepsy trials. They did simulations with around
1000 patients.

\citep{qi2019predicting} propose a framework to predict the Phase 3
clinical trial results based on the Phase 2 clinical trial data. They
proposed an individual treatment effect model to model the relationship
between patient baseline characteristics, Ctrough and clinical
outcomes.

\section{Neural Network + Survival model Literature Review}

Summarize two papers \citep{katzman2018deepsurv} and
\citep{hu2023conditional}. In particular, we read the code of
\citep{hu2023conditional} and summarize the structure of their neural
network.

\subsection{Notations}

For subject $i$, denote the time-varying covariate vector as
$X_{i}(t)$, the failure time as $T_i$, the censoring time as $C_i$,
observed time $Y_i=\min \left( T_i, C_i \right)$ and the censoring
indicator $\Delta_i=I(T_i \leq C_i)$. We have $n$ i.i.d. observations
$\{Y_i, \Delta_i, \tilde{X}_i(\cdot): i=1, \ldots, n\}$ where
$\tilde{X}_i(t)=\left\{X_i(s), 0 \leq\right.$ $s \leq t\}$ denotes the
covariate history up to time $t$.

Let $f(t \mid \tilde{X}_i(\infty))$ and $f_{C}(t \mid
\tilde{X}_i(\infty))$ be the conditional density function of $T_i$ and
$C_i$, $S(t \mid \tilde{X}_i(\infty))$ and $S_{C}(t \mid
\tilde{X}_i(\infty))$ be the conditional survival function of $T_i$
and $C_i$.  Let the conditional hazard function of $T_i$ be $\lambda
\left[t \mid \tilde{X}_i(\infty)\right]$. Assume the hazard only
depends on the current values of the covariate, that is, $\lambda
\left[t \mid \tilde{X}_i(\infty)\right] = \lambda \left[t \mid
\tilde{X}_i(t)\right] = \lambda \left[t \mid X_i(t)\right]$. Define
$h(t,X_i(t))=\log (\lambda (t \mid X_i(t)))$. The conditional survival
function is given by

\begin{equation} S\left[t \mid \widetilde{X}_i(\infty)\right]=S\left[t
\mid \widetilde{X}_i(t)\right]=\exp \left\{-\int_0^t e^{h\left[s,
X_i(s)\right]} d s\right\}
\label{eq:survfunc}
\end{equation}

Assume censoring time is independent of failure time conditional on
covariates. Then given the observed data $\{y_i, \delta_i, x_i(\cdot):
i=1, \ldots, n\}$, the log likelihood function is

\begin{equation} \ell_n=\sum_{i=1}^n\left\{h\left[y_i,
x_i\left(y_i\right)\right] \delta_i - \int_0^{y_i} e^{h\left[t,
x_i(t)\right]} d t\right\}.
\label{eq:likelihood}
\end{equation}

Using $h$ instead of $\lambda$ removes the positivity constraint for
$\lambda$, hence simplifies the optimization algorithm for estimating
(\ref{eq:likelihood}). Furthermore, (\ref{eq:survfunc}) is always a
valid survival function for any function $h$.

When fitting a survival model with time-varying covariates, the data
set is usually expanded and each row is treated as an
observation. $(t_1, \ldots, t_n)$ are sorted values of $(y_1, \ldots,
y_n)$ and $\delta_{ij} = \delta_{i} I(t_j = y_i)$.

\begin{table}[h]
\centering
\begin{tabular}{cccccc}
\hline
$i$ & start time & stop time & $\delta_{ij}$ & covariates \\ 
\hline
1 & $t_0 = 0$ & $t_1$ & 0 & $x_1(t_1)$ \\
1 & $t_1$ & $t_2$ & 0 & $x_1(t_2)$ \\
\vdots & \vdots & \vdots & \vdots & \vdots \\
1 & $\dots$ & $y_1$ & $\delta_1$ & $x_1(y_1)$ \\
2 & $t_0 = 0$ & $t_1$ & 0 & $x_2(t_1)$ \\
2 & $t_1$ & $t_2$ & 0 & $x_2(t_2)$ \\
\vdots & \vdots & \vdots & \vdots & \vdots \\
2 & $\dots$ & $y_2$ & $\delta_2$ & $x_2(y_2)$ \\
\vdots & \vdots & \vdots & \vdots & \vdots \\
\hline
\end{tabular}
\caption{New Data Structure}
\label{tab:label1}
\end{table}

Now maximizing the log likelihood function (\ref{eq:likelihood}) is
equivalent to minimizing the following loss function

$$L(h)=\frac{1}{n} \sum_{i=1}^n \sum_{j=1}^n I\left(t_j \leq y_i\right)\left\{e^{h\left[t_j, x_i\left(t_j\right)\right]}\left(t_j-t_{j-1}\right)-h\left[t_j, x_i\left(t_j\right)\right] \delta_{i j}\right\}.$$
Once $\hat{h}$ is obtained using neural networks, the conditional
survival curve can be estimated by plugging $\hat{h}$ into
(\ref{eq:survfunc}).

\subsection{Neural Networks}

Now we discuss how the neural network works. In this paper, the
network has two hidden layers($l=1,2$) and one output
layer($l=3$). The structure is

\begin{align*} a^{(1)} &= \sigma(W^{(1)} X + b^{(1)}) \\ a^{(2)} &=
\sigma(W^{(2)} a^{(1)} + b^{(2)}) \\ a^{(3)} &= W^{(3)} a^{(2)} +
b^{(3)} = h \\
\end{align*} where $X$ is the input features vector,
$\sigma(z)=\max(0,z)$ is the ReLU activation function. The output
layer has a linear activation function.  $W$ and $b$ are parameters to
be trained, but the dimensions of $W$ and $b$ (hidden layers nodes)
are hyperparameters. For example, if the dimension of features vector
is $5$ and hidden layers nodes is chosen to be $64$, then the
dimension of $W^{(1)}$ is $64 \times 5$ and $b^{(1)}$ is $64$,
$W^{(2)}$ is $64 \times 64$ and $b^{(1)}$ is $64$, $W^{(3)}$ is $1
\times 64$ and $b^{(3)}$ is $1$. There are $4609$ parameters to be
trained. Denote all the parameters to be trained as $\theta$. Then
$L(h)$ is actually $L(\theta)$.

Next we compute the gradient of the loss function with respect to each
parameter. The Python deep learning platform can automatically compute
the gradient based on a customized loss function. After calculating
the gradients, we update the parameters in the direction that reduces
the loss.
$$\theta_{1} = \theta_{0} - \alpha \cdot \nabla_\theta L$$
where $\alpha$ is called the learning rate, $\nabla_\theta L$ is the
gradient of the loss function $L$ with respect to $\theta$. This is
done using an optimization algorithm(optimizer) like Stochastic
Gradient Descent (SGD) or Adaptive Moment Estimation (Adam).

However, when we train a neural network with a deep learning framework
in Python, the loss function for a neural network model typically
requires at least two arguments: the true value (ground truth) and the
output from the neural network. What we're doing is fitting a function
with no true value. In their code, they take $\delta_{ij}$ to be the
true value and define the loss function in this way:

$$L(h)=\frac{1}{n} \sum_{i=1}^n \left\{ \sum_{j:{t_j \leq y_i}} \left\{e^{h\left[x_i\left(t_j\right)\right]}\Delta(t_j)-h\left[x_i\left(t_j\right)\right] \delta_{i j}\right\} \right\}.$$

The paper mentioned that 'The input of neural networks is $\left[
t_{j-1}, t_j, x_i(t_j)\right]$'. More precisely, $t_{j-1}$ and $t_j$
are used to compute $\Delta(t_j)$ in the loss and $x_i(t_j)$ are used
in the neural network.

Dividing the full data into batches is another important trick in
neural network training. Recall that $L(h)=\frac{1}{n} \sum_{i=1}^n
L_{i}(\theta)$ and $\theta_{1} = \theta_{0} - \alpha \cdot
\sum_{i=1}^n \nabla_\theta L_i$. To economize on the computational
cost when $n$ is large, we divide the full data into $\{B_1, ...,
B_k\}$. We update $\theta$ based on the gradient sum over $B_1$, then
update $\theta$ based on the gradient sum over $B_2$ and so on. If go
through the full data is called an epoch, we now divide an epoch into
k iterations. Usually the data size is the same for each batch.

\subsection{Simulation and Application}

Simulation results show that that when the Cox model is correctly
specified, both the partial likelihood method and the proposed neural
networks method perform well. When the Cox model is misspecified, the
partial likelihood method yields severe biases, whereas the proposed
neural network method still works well with a similar performance.

They use four real-world data sets with sample sizes of a few thousand
and the covariate numbers range from 7 to 14. The covariates in these
data sets are all time independent.

A common criterion is the C-index which estimates the probability that
the predicted survival times of two comparable individuals have the
same ordering as their true survival times.

$$
\text { C-index }=P\left[\hat{S}\left(T_i \mid
x_i\right)<\hat{S}\left(T_i \mid x_j\right) \mid T_i<T_j,
\Delta_i=1\right].
$$

\subsection{Discussion}

\begin{itemize}
  \item This paper is only for low dimensional inputs. If the
dimension is larger than the sample size, it may be necessary to use
regularization methods.
  \item The data expansion would increase the effective sample size
from $n$ to $n^2$.
  \item These neural network survival models are predicting hazard
functions or survival functions. they don't predict survival time
directly.
\end{itemize}


\section{Sample Size Re-estimation Literature Review}

Summarize a review paper \citep{Liu2021review}'Sample size
re-estimation for pivotal clinical trials'.

\subsection{Notations}

At the design stage, we set the total number of patients as
$n_2$. They are randomized with 1:1 ratio to receive treatment or
control.

Denote observations as $X_{ij}$, $i=R,C$ and $j=1,...,n_{2}/2$. Assume
$X_{ij} \sim N(\mu_{i}, \sigma^{2})$.

Define the true treatment effect as $\theta=\mu_R-\mu_C$. The null
hypothesis is $H_0 = \theta \leq 0$ vs. $H_a = \theta > 0$.

At final analysis(FA), the final test statistic is $Z_2 =
\frac{\hat{\theta}_2}{\sqrt{4\sigma^2/n_2}}$, where $\hat{\theta}_2 =
\frac{\sum_{j=1}^{n_2/2} X_{Rj}}{n_2/2} - \frac{\sum_{j=1}^{n_2/2}
X_{Cj}}{n_2/2}$ and $\hat{\theta}_2 \sim N(\theta,
\frac{4\sigma^2}{n_2})$. We need to determine sample size $n_2$ at the
design stage. Given the one-side Type 1 error $\alpha$, power
$1-\beta$ and the assumed treatment effect $\theta_d$, we can
calculate the total sample size as follows:

$$n_2=4\sigma^2(\frac{z_{\alpha}+z_{\beta}}{\theta_d})^2,$$
where $z_{\alpha} = \Phi^{-1}(1 - \alpha)$. It comes from
$\frac{\theta_d}{\sqrt{4\sigma^2/n_2}}=z_{\alpha}+z_{\beta}$.

Interim analysis is conducted when we obtain $n_1$ samples. Define
information fraction as $t=n_1/n_2$. At interim analysis, we have
interim test statistic $Z_1 =
\frac{\hat{\theta}_1}{\sqrt{4\sigma^2/n_1}}$ where $\hat{\theta}_1 =
\frac{\sum_{j=1}^{n_1/2} X_{Rj}}{n_1/2} - \frac{\sum_{j=1}^{n_1/2}
X_{Cj}}{n_1/2}$ and $\hat{\theta}_1 \sim N(\theta,
\frac{4\sigma^2}{n_1})$.

Define incremental sample size $\tilde{n}_2=n_2-n_1$. Incremental test
statistic is $\tilde{Z}_2 =
\frac{\hat{\tilde{\theta}}_2}{\sqrt{4\sigma^2/\tilde{n}_2}}$ where
$\hat{\tilde{\theta}}_2 = \frac{\sum_{j=(n_1/2)+1}^{n_2/2}
X_{Rj}}{\tilde{n}_2/2} - \frac{\sum_{j=(n_1/2)+1}^{n_2/2}
X_{Cj}}{\tilde{n}_2/2}$ and $\hat{\tilde{\theta}}_2 \sim N(\theta,
\frac{4\sigma^2}{\tilde{n}_2})$. $Z_2$ can be represented by $Z_1$ and
$\tilde{Z}_2$

$$Z_2 = \sqrt{t}Z_1 + \sqrt{1-t}\tilde{Z}_2.$$

After sample size re-etimation(SSR), we have the adapted total sample
size as $n_{2}^{*}$ and incremental sample size
$\tilde{n}_{2}^{*}$. Similarly, define $\tilde{Z}_{2}^{*} =
\frac{\hat{\tilde{\theta}}_{2}^{*}}{\sqrt{4\sigma^2/\tilde{n}_{2}^{*}}}$
where $\hat{\tilde{\theta}}_{2}^{*} =
\frac{\sum_{j=(n_1/2)+1}^{n_{2}^{*}/2} X_{Rj}}{\tilde{n}_{2}^{*}/2} -
\frac{\sum_{j=(n_1/2)+1}^{n_{2}^{*}/2}
X_{Cj}}{\tilde{n}_{2}^{*}/2}$. Then $Z_{2}^{*}$ can be represented by
$Z_1$ and $\tilde{Z}_{2}^{*}$ as $Z_{2}^{*} =
\sqrt{\frac{n_1}{n_{2}^{*}}}Z_1 +
\sqrt{1-\frac{n_1}{n_{2}^{*}}}\tilde{Z}_{2}^{*}$.

One problem of conventional final analysis is that if the rejection
rule for $H_0$ at FA is left unchanged, i.e. $Z_{2}^{*}>z_{\alpha}$,
the Type I error rate can be inflated,
i.e.$P_{H_0}(Z_{2}^{*}>z_{\alpha})>\alpha$. Many methods have been
proposed to control the Type I error and will be mentioned in Section
2.4. For example, a popular method is replace $Z_{2}^{*}$ by
$Z_{2,chw}^{*} = \sqrt{\frac{n_1}{n_{2}}}Z_1 +
\sqrt{\frac{\tilde{n}_2}{n_2}}\tilde{Z}_{2}^{*}$. In this way,
$P_{H_0}(Z_{2,chw}^{*}>z_{\alpha})=\alpha$.


\subsection{Conditional power}

Next we define conditional power(CP, probability of rejecting null
hypothesis at final analysis given the interim test statistic
$Z_1=z_1$). It is defined as

\begin{align*} CP(\theta, z_1, n_2, t) &= P_{\theta}(Z_2 > z_{\alpha}
| Z_1 = z_1) \\ &= P_{\theta}\left(\sqrt{\frac{n_1}{n_2}} z_1 +
\sqrt{\frac{\tilde{n}_2}{n_2}} \tilde{Z}_2 > z_{\alpha}\right) \\ &=
P\left(\frac{\hat{\tilde{\theta}}_2 - \theta}{\sqrt{4\sigma^2 /
\tilde{n}_2}} > \sqrt{\frac{\tilde{n}_2}{n_2}} z_{\alpha} -
\sqrt{\frac{n_1}{\tilde{n}_2}} z_1 - \frac{\theta}{\sqrt{4\sigma^2 /
\tilde{n}_2}}\right) \\ &= 1 -
\Phi\left(\sqrt{\frac{\tilde{n}_2}{n_2}} z_{\alpha} -
\sqrt{\frac{n_1}{\tilde{n}_2}} z_1 - \frac{\theta}{\sqrt{4\sigma^2 /
\tilde{n}_2}}\right) \\ &= \Phi\left(\frac{\theta
\sqrt{n_2(1-t)}}{2\sigma} - \frac{z_{\alpha} - z_1
\sqrt{t}}{\sqrt{1-t}}\right).
\end{align*}

Since $\theta$ is unknown, we can replace it with the assumption
treatment effect at the design stage $\theta_d$ or the interim
observed treatment effect estimator $\hat{\theta}_1$.

When using $\hat{\theta}_1$, $CP(\hat{\theta}_1, z_1, n_2,
t)=\Phi\left( \frac{z_1}{\sqrt{t(1-t)}} -
\frac{z_{\alpha}}{\sqrt{1-t}}\right)$. When using $\theta_d$,
$CP(\theta_d, z_1, n_2, t)=\Phi\left( \frac{\theta_d
\sqrt{n_2(1-t)}}{2\sigma} - \frac{z_{\alpha}-z_1
\sqrt{t}}{\sqrt{1-t}}\right)$.

\subsection{Sample size adaptation rule}

The goal of the interim analysis is to determine the new sample size
for the trial $n_{2}^{*}$. Typically $n_{2}^{*}$ should be larger than
$n_2$ and threshold by a maximum value $n_{max}$. A basic idea is to
keep $CP(\theta, z_1, n_{2}^{*}, t)=P_{\theta}(Z_{2}^{*} > z_{\alpha}
| Z_1 = z_1)$ at $1-\beta$ when the interim result is promising.


Now we refer to (Mehta, 2010) to introduce how to determine promising
zone based on $CP(\theta, z_1, n_{2}^{*}, t)$.

There are many other methods to determine $n_{2}^{*}$. See section 2.3
in \citep{Liu2021review}.





\bibliographystyle{mcap}

\bibliography{refs}

\end{document}