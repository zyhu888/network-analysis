\documentclass[12pt]{article}

%% preamble: Keep it clean; only include those you need
\usepackage{amsmath}
\usepackage[margin = 1in]{geometry}
\usepackage{graphicx}
\usepackage{booktabs}
\usepackage{natbib}

%% for double spacing
\usepackage{setspace}

% for space filling
\usepackage{lipsum}
% highlighting hyper links
\usepackage[colorlinks=true, citecolor=blue]{hyperref}

\usepackage{indentfirst}

%% meta data

\title{Network analysis}
\author{Zeyu Hu\\
}

\begin{document}
\maketitle

\doublespacing

\section{Introdcution}
Understanding collaboration networks in scientific research is crucial for advancing knowledge and fostering innovation. In the field of public health, the study of collaboration patterns can reveal important insights into how research is conducted, how knowledge is disseminated, and how professional relationships influence scientific outcomes. 

Assortativity, a measure of the tendency of nodes in a network to connect with similiar or dissimilar nodes based on certain attributes, has been widely studied across various fields, including healthcare. \citep{newman2002assortative} explored assortativity in various types of networks, finding that social networks tend to be assortatively mixed by attributes such as age, profession, or socio-economic status. \citep{moody2004structure} studied a social science collaboration network and shows high assortativity by discipline, suggesting that similar professional backgrounds foster closer collaborations. \citep{holme2006nonequilibrium} discussed how assortativity affects the coevolution of network structure and node attributes, such as opinions or behaviors. They found that networks with high assortativity tend to maintain their structure more stably over time. \citep{guimera2005team} demonstrate that assortativity based on expertise within teams leads to higher performance in scientific collaborations. \citep{piraveenan2010assortative} developed approaches to measure local assortativity in complex networks, which allows for a more detailed analysis of how individual nodes contribute to the overall assortativity of the network.

Efficiency, a measure of how efficiently the network exchanges information. On a global scale, efficiency quantifies the exchange of information across the whole network where information is concurrently exchanged. \citep{latora2001efficient} introduced global efficiency as a measure of a network’s performance in terms of information transfer, highlighting its significance in networks with a shorter average path length. \citep{achard2007efficiency} explored the relationship between efficiency and cost in networked systems, particularly focusing on brain networks. Their findings have been applied to healthcare networks, demonstrating how trade-offs between efficiency and network costs can influence the overall performance of healthcare systems. \citep{bullmore2009complex} examined efficiency within the context of brain connectivity networks, where they applied graph theoretical measures to analyze how efficiently brain regions communicate with each other. They found that high global efficiency in these networks is linked to better cognitive functioning, offering a detailed exploration of efficiency as a core attribute of complex systems.

The clustering coefficient is a measure of the degree to which nodes in a network tend to cluster together. In healthcare networks, this metric is often used to assess how doctors or institutions form tightly-knit groups, which can have implications for the speed and effectiveness of information dissemination, collaboration, and overall network resilience. \citep{watts1998collective}  introduced the concept of the clustering coefficient in the context of “small-world” networks, where they found that many real-world networks, including social and biological networks, exhibit a high clustering coefficient. This foundational work highlights how nodes in such networks tend to form dense clusters, which can facilitate rapid information sharing within communities. \citep{newman2001structure} applied the clustering coefficient to scientific collaboration networks, finding that researchers tend to collaborate in tightly-knit groups, which is indicative of the broader trend in professional networks where close connections are common. \citep{onnela2007structure} explored the structure of mobile communication networks and observed that they have high clustering coefficients. \citep{boccaletti2006complex} reviewed the structure and dynamics of complex networks, focusing on the clustering coefficient as a key measure of network topology. Their analysis of various types of networks, including social and technological networks, provides insights into how clustering contributes to the robustness and efficiency of these systems.

In our study, we conducted a comprehensive analysis of key network metrics, including assortativity, efficiency, and clustering coefficient, to understand the structural and functional characteristics of healthcare networks. For each of these metrics, we calculated the overall values across the entire network and assessed their confidence intervals to ensure the robustness of our findings.Additionally, we examined how these metrics varied within sub-networks of particular interest, including certain institutions and specialties, to uncover how these patterns vary across different contexts.

\section{Assortativity}
\subsection{Overall Network Assortativity}

The assortativity of a network measures the tendency of nodes to connect with similar nodes. For our collaboration network of otolaryngologists in the United States, the overall assortativity coefficient was calculated to be 0.42. This positive value indicates a moderate tendency for otolaryngologists to collaborate with others who have similar attributes. The calculation of assortativity was performed using the wdnet package, as detailed in \citep{yuan2021assortativity} and \citep{yuan2023generating}


To assess the robustness of this result, according to \citep{pigorsch2022assortative}, we employed the jack-knife method, which involves sequentially removing each vertex from the network and recalculating the assortativity coefficient. This resampling technique provided a 95\% confidence interval for the assortativity coefficient, ranging from 0.32 to 0.53. This interval suggests that the network's assortativity is significantly different from zero, reinforcing the presence of assortative mixing in our network. 
\subsection{Subnet Analysis by Institution}

To delve deeper into the network’s structure, we were particularly interested in analyzing specific subsets of the overall network. We decided to use institutions as a criterion for this subset analysis, focusing on the top five institutions by the number of otolaryngologists. For each institution, we studied the assortativity of the subgraph consisting of all nodes belonging to that institution to understand the collaboration patterns among otolaryngologists within the same institution. The assortativity coefficients for these institutions were as follows:

\begin{itemize}
    \item \textbf{Icahn School of Medicine at Mount Sinai/New York Eye and Ear Infirmary at Mount Sinai}: 0.11
    \item \textbf{Vanderbilt University Medical Center}: 0.23
    \item \textbf{Montefiore Medical Center/Albert Einstein College of Medicine}: 0.48
    \item \textbf{University of Michigan Health System}: 0.33
    \item \textbf{University of Southern California/LAC+USC Medical Center}: 0.18
\end{itemize}

These results indicate varying degrees of assortativity across different institutions. For instance, Montefiore Medical Center/Albert Einstein College of Medicine showed the highest assortativity coefficient (0.48), suggesting a stronger tendency for otolaryngologists within this institution to collaborate with each other. In contrast, the Icahn School of Medicine at Mount Sinai/New York Eye and Ear Infirmary at Mount Sinai exhibited a lower assortativity coefficient (0.11), indicating a more diverse pattern of collaborations within the institution.

\subsection{Subnet Analysis by Specialty}

In addition to institutional analysis, we also focused on the subnetwork by specialty, recognizing that collaborations might differ significantly based on specific fields within otolaryngology. By examining the top five most represented specialties, we aimed to understand how specialty influences collaboration patterns. The assortativity coefficients for these subnets were:

\begin{itemize}
    \item \textbf{Head \& Neck}: 0.47
    \item \textbf{Pediatric}: 0.28
    \item \textbf{General}: 0.65
    \item \textbf{Neurotology}: 0.40
    \item \textbf{Facial Plastics}: 0.56
\end{itemize}

The highest assortativity coefficient was observed for the General specialty (0.65), indicating a strong tendency for general otolaryngologists to collaborate with each other. This high value suggests that otolaryngologists within the General specialty frequently form collaborations within their group, possibly due to shared research interests and goals. Conversely, the Pediatric specialty had the lowest assortativity coefficient (0.28) among the top five specialties, suggesting more diverse collaborations within this group. This lower value indicates that Pediatric otolaryngologists are more likely to collaborate outside their immediate specialty, reflecting a broader range of interactions and interdisciplinary work.


\bibliographystyle{mcap}

\bibliography{refs}

\end{document}